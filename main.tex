\documentclass[]{NTUTCSIEproject}
\usepackage{amsmath}

\addbibresource{references.bib}
\def\projectTitle{實務專題計畫期中摘要報告 撰寫格式說明(更改為專題題目)}
\def\projectNumber{113-CSIE-SXXX-MID}
\def\projectPeriod{112 年第 1 學期至 113 年第 1 學期}
\def\projectInstructor{OOO}
\newcommand{\member}[1]{&#1\\}
\def\projectMembers{
    \member{xxxxxx(學號) OOO(姓名)}
    \member{xxxxxx(學號) OOO(姓名)}
    \member{xxxxxx(學號) OOO(姓名)}
}

\newcommand\makeptitle{
{\begin{center}
    {\fontsize{16pt}{1em}\selectfont\textbf{\projectTitle}}
    \vspace*{11pt}\vspace*{0pt}
    \hspace*{1.65in}
    \begin{minipage}{\linewidth}
        \begin{flushleft}
        {\fontsize{14pt}{1em}\selectfont
        專題編號:\projectNumber \\
        執行期限:\projectPeriod \\
        指導教授:\projectInstructor \\
        \begin{tabular}{@{}l@{}l}
        專題參與人員:\ \projectMembers
        \end{tabular}}
        \end{flushleft}
    \end{minipage}
    \vspace*{11pt}\vspace*{0pt}
\end{center}}}
 % makeptitle

\begin{document}

\pagenumbering{arabic}

\setlength{\columnsep}{0.5cm}
\twocolumn[\makeptitle]

\section*{一、摘要}

摘要為論文或報告的精簡概要,其目的是透過簡短的敘述,使讀者大致瞭解整篇報告的內容。摘要的內容通常須包括問題的描述,使用方法以及所得到的結果,以不超過200\textasciitilde300字為原則。
\\\textbf{關鍵詞:列舉3\textasciitilde6個描述重點的專有名詞。}

\section*{二、緣由及目的}

資工系實務專題競賽各參與者,需依此格式繕寫期中進度摘要報告及競賽成果摘要報告。實務專題成果報告請另依本校圖書館學位論文格式規範撰寫

\section*{三、研究報告內容}

此摘要報告的內容,除題目及專題參與人員資料以外,依序應包含作品的特性,如計畫緣由與目的、研究範圍、使用技術方法、架構流程、工具說明、實驗結果、結論、參考文獻等,期中報告則以預計進行方式及預期成果為主,篇幅以兩頁為限。

\section*{四、格式注意事項}

此摘要報告撰寫格式,說明如下:

\subsection*{(一) 用紙}

使用A4紙,即29.7公分×21公分。

\subsection*{(二) 版面段落格式}

中文打字規格為單行繕打(行間不另留間距),英文打字規格為Single Space。但在本文與各章節標題之間,請隔一行繕打。中文次標題依序為:一、(一)、1、(1)。
繕打時採用橫式,除題目與參與人員者資料採一欄,置中對齊外,其他分兩欄,採左右對齊。每頁上下側及左右邊各留2.5公分,每欄的寬度是7.75公分,而在兩欄間相隔0.5公分。

\subsection*{(三) 字體}

報告的正文以中文撰寫。在字體的使用方面,英文使用Times New Roman Font,中文使用標楷體,字體大小請以12號為主。

\subsection*{(四) 頁碼}

頁碼的編寫,請以阿拉伯數字依順序標記在每頁下方中央。

\subsection*{(五) 圖表}

為便於讀者閱讀,請盡可能將圖表置於出現圖表說明的該段文字之後。比較大的圖表,可以含跨兩個欄。各圖表請備說明內容,圖的說明應置於圖的下方(如圖1. xxx),而表的說明則應置於表的上方。

\subsection*{(六) 數學符號及方程式}

請置於欄位之中央位置並以小括號編號(1) (2) …等,數學符號一律使用斜體字體。
\begin{align}
y=Ax+b
\end{align}
\subsection*{(七) 參考文獻}

參考文獻請參考IEEE或本校學位論文參考文獻格式規範撰寫,請依參照使用之順序,依次編號列出。\cite{englishExample}\cite{chineseExample}

\printbibliography[title=參考文獻]

\end{document}
